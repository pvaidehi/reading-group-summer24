\documentclass[11pt,notes=hide,aspectratio=169,mathserif]{beamer}

% PACKAGES
\usepackage{graphics}  % Support for images/figures
\usepackage{graphicx}  % Includes the \resizebox command
\usepackage{url}	   % Includes \urldef and \url commands
\usepackage{natbib}
\usepackage{bibentry}  % Includes the \nobibliography command
\usepackage{verbatim}  %Supports comments
\usepackage{booktabs} %Supports \toprule, \bottomrule, etc in tables
\usepackage{etoolbox}  %Supports toggle commands
\usepackage{datetime}
\usepackage{bm}	%Supports bold math \bm

% PACKAGES (that should already be included by your LyX document settings)
\usepackage{amsfonts}  % Lots of stuff, including \mathbb
\usepackage{amsmath}   % Standard math package
\usepackage{amsthm}    % Includes the comment functions
\usepackage[utf8]{inputenc}
\usepackage[T1]{fontenc}

% CUSTOM DEFINITIONS
\def\newblock{} %Get beamer to cooperate with BibTeX
\linespread{1.2}

% IDENTIFYING INFORMATION
\title[reading group summer 24]{reading group summer 24 \\ Location Sorting and Endogenous Amenities: Evidence from Amsterdam (2024)}
\author[Vaidehi \& friends ]{Milena Almagro and Tomas Domınguez-Iino}
\date{\monthname[\the\month] \the\year}

% THEMATIC OPTIONS
\setbeamercovered{transparent}
\usetheme{metropolis}
\usecolortheme[snowy]{owl}
\beamertemplatenavigationsymbolsempty
\setbeamertemplate{footline}[] % Remove the footline with page number

% BACKUP SLIDE NUMBERING
\usepackage{appendixnumberbeamer}


\AtBeginSection[]{
  \begin{frame}{Outline}
    \tableofcontents[currentsection]
  \end{frame}
}

\begin{document}

%---------------------------------------------------------------------
\begin{frame}[plain]
\titlepage
\note{
	\begin{itemize}
	\end{itemize}
}
\end{frame}
%---------------------------------------------------------------------

\section{Introduction}
%---------------------------------------------------------------------
\begin{frame}{Introduction}
	\begin{itemize}
		\item hello
	\end{itemize}
\end{frame}
%---------------------------------------------------------------------

%---------------------------------------------------------------------
\begin{frame}{Context}
	\begin{itemize}
        \item massive expansion in tourism in Amsterdam
		\item increased supply of private rentals, increased supply of STRs
        \item New regulation in Amsterdam severely restricting STR supply (hotels and Airbnbs)
       
	\end{itemize}
\end{frame}
%---------------------------------------------------------------------

%---------------------------------------------------------------------
\begin{frame}{Related Literature}
	\begin{itemize}
		\item spatial equilirium models
        \item effects of STR entry on the housing market and hotel revenue
        \item discrete-choice tools from the empirical io literature applied to urban residential markets
	\end{itemize}
\end{frame}
%---------------------------------------------------------------------

\section{Data}
%---------------------------------------------------------------------
\begin{frame}{Data}
	\begin{itemize}
		\item Two datasets
        \item First, HAMP data: loan-level dataset includes information on borrower characteristics and mortgage terms before and after modification
        \item it also includes the expected gain calculation run by servicers when evaluating borrowers for each modification type
        \item match to consumer credit bureau records from TransUnion
        \item Second, Chase Bank data: account-level monthly information on all mortgages serviced by Chase Bank and spending by mortgagors who also had a Chase credit card
        \item ncludes all borrowers who receive either a government-subsidized HAMP or private modification
	\end{itemize}
\end{frame}
%---------------------------------------------------------------------


\section{Effect of Principal Reduction on Default}

%---------------------------------------------------------------------
\begin{frame}{Background on HAMP}
    \begin{itemize}
         \item introduced in 2009 in response to foreclosure crisis
         \item provided government subsidies to help facilitate mortgage modifications for borrowers struggling to make their payments
         \item primary goal of HAMP modifications is to provide borrowers with more affordable mortgages
         \item all borrowers who receive modifications have their payment reduced to reach a 31 percent payment-to-income (PTI) ratio for at least five years
         \item set of elgibility criteria 
         \item mean payment reduction is \$680 per month
    \end{itemize}
 \end{frame}
 %---------------------------------------------------------------------

%---------------------------------------------------------------------
\begin{frame}{Background on HAMP}
\begin{itemize}
        \item contrasting borrowers assigned to two distinct modification types
        \item “payment reduction” modification vs “payment and principal reduction” modification
        \begin{figure}
        \centering
        \includegraphics[height=0.3\textwidth]{~/Desktop/Screenshot 2024-08-14 at 2.23.11 PM.png}
        \caption{}
        \label{fig:photo}
    \end{figure}
\end{itemize}
\end{frame}
%---------------------------------------------------------------------

%---------------------------------------------------------------------
\begin{frame}{Identification}
\begin{itemize}
        \item based on cutoff rule (CUTOFF $\implies$ RDD!!)
        \item servicers calculate the expected NPV of cash flows for lenders under the status quo and under each of the two modification types
        \item principal reduction is determined in part by a calculation examining which modification type is expected to be most beneficial for the lender
        \begin{figure}
        \centering
        \subfigure[Figure 1]{\includegraphics[height=0.1\textwidth]{~/Desktop/Screenshot 2024-08-14 at 2.50.51 PM.png}}
        \hspace{1cm}
        \subfigure[Figure 2]{\includegraphics[height=0.1\textwidth]{~/Desktop/Screenshot 2024-08-14 at 2.50.57 PM.png}}
    \end{figure}
\end{itemize}
\end{frame}
%---------------------------------------------------------------------

%---------------------------------------------------------------------
\begin{frame}{Results}
\begin{itemize}
        \item principal reduction has no impact on default
        \item principal reduction was also costly to lenders: estimate that they had to forgive at least \$1.3 million in principal to prevent one foreclosure
        \item government spent about \$8,000 per modification to support the additional principal reduction of the size we analyze in our treatment group. This translates into a cost of at least \$365,000 per avoided foreclosure, almost an order of magnitude larger than common estimates of the social costs of foreclosure
        \item inconsistent with prior evidence which relied on cross-sectional evidence
\end{itemize}
\end{frame}
%---------------------------------------------------------------------

\section{Effect of Principal Reduction on Consumption}

%---------------------------------------------------------------------
\begin{frame}{Identification}
    \begin{itemize}
        \item diff-in-diff design
        \item control group: set of underwater borrowers who were eligible for principal reductions, but who instead received only payment reduction modifications
        \item treatment captures the effect of long-term debt forgiveness holding short-term payments and access to liquidity fixed
        \begin{figure}
            \centering
            \includegraphics[height=0.05\textwidth]{~/Desktop/Screenshot 2024-08-14 at 3.31.02 PM.png}
            \caption{parallel trends}
            \label{fig:photo}
        \end{figure}
        \item result: no effect on consumption
    \end{itemize}
\end{frame}
%---------------------------------------------------------------------

\section{Effect of Payment Reduction on Default}

%---------------------------------------------------------------------
\begin{frame}{Background}
    \begin{itemize}
        \item compare HAMP to private modifications
        \item private modifications used a payment reduction target (vs PTI) - larger decreases in payments
        \item they use maturity extension as a low-cost tool for achieving deeper immediate payment reductions without reducing long-term obligations
        \begin{figure}
            \centering
            \subfigure[Figure 1]{\includegraphics[width=0.3\textwidth]{~/Desktop/Screenshot 2024-08-14 at 4.13.42 PM.png}}
            \hspace{1cm}
            \subfigure[Figure 2]{\includegraphics[width=0.3\textwidth]{~/Desktop/Screenshot 2024-08-14 at 4.14.15 PM.png}}
        \end{figure}
    \end{itemize}
\end{frame}
%---------------------------------------------------------------------

%---------------------------------------------------------------------
\begin{frame}{Identification}
    \begin{itemize}
        \item used HAMP elgibility cutoff of 31\% PTI
        \item those below cutoff: only private modifications; above the cutoff: half and half
        \item PTI cutoff therefore serves as an instrument for allocating borrowers between HAMP modifications with small payment reductions and private modifications with large payment reductions
    \end{itemize}
\end{frame}
%---------------------------------------------------------------------

%---------------------------------------------------------------------
\begin{frame}{Identification}
    \begin{itemize}
        \item payment reductions are approximately constant below the cutoff and increasing above the cutoff
        \item the default rate falls sharply by 7.3 percentage points
        \begin{figure}
            \centering
            \subfigure[Figure 1]{\includegraphics[height=0.3\textwidth]{~/Desktop/Screenshot 2024-08-14 at 4.19.07 PM.png}}
            \hspace{1cm}
            \subfigure[Figure 2]{\includegraphics[height=0.3\textwidth]{~/Desktop/Screenshot 2024-08-14 at 4.19.25 PM.png}}
        \end{figure}
    \end{itemize}
\end{frame}
%---------------------------------------------------------------------

\section{Discussion \& Conclusion}
%---------------------------------------------------------------------
\begin{frame}{Dicussion: default}
    \begin{itemize}
        \item key result: liquidity drives default and negative equity does not affect default
        \item time frame matters: this treatment was implemented in 2010
        \item more generous principal reductions could affect default rates
        \item principal reduction is ineffective for borrowers and costly to both lenders and taxpayers
        \item payment-focused modifications are able to successfully reduce defaults for borrowers, at zero cost to taxpayers and at negative cost to lenders
    \end{itemize}
\end{frame}
%---------------------------------------------------------------------

%---------------------------------------------------------------------
\begin{frame}{Dicussion: consumption}
    \begin{itemize}
        \item two key channels for relationship between housing wealth and consumption: wealth and collateral constraints
        \item MPC from principal reduction is effectively zero ...
        \item ... suggesting that the wealth channel is weak
        \item the timing of liquidity matters
        \item show that current consumption is unresponsive to changes in future liquidity
    \end{itemize}
\end{frame}
%---------------------------------------------------------------------

%---------------------------------------------------------------------
\begin{frame}{Conclusion}
	\begin{itemize}
        \item applied to the main government program for distressed borrowers during the Great Recession, our results imply that 267,000 defaults could have been avoided
		\item i liked this paper but some parts were way too boring
        \item the appendix had a bunch of models ???
	\end{itemize}
\end{frame}
%---------------------------------------------------------------------

%---------------------------------------------------------------------
\begin{frame}[plain]
\begin{center}{\LARGE See ya}\end{center}
\end{frame}
%---------------------------------------------------------------------


% %\beginbackup
% %\appendix
% %\input{sections/appendix.tex}

% %\bibliographystyle{../bib/aeanobold}
% %\nobibliography{bib.bib}
% %\backupend
\end{document}